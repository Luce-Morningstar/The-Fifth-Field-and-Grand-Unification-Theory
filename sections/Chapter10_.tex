\chapter{Formalism and Equations}
\renewcommand{\thesection}{10.\arabic{section}}

\section{Projection and State Validity}

\begin{equation}
\hat{P}^2 = \hat{P}, \quad \hat{P}^\dagger = \hat{P}
\end{equation}

If $\hat{P} \Psi = \Psi$, then the state passes the observational filter and is allowed to exist~\cite{wheeler_it_from_bit}.

If $\hat{P} \Psi = 0$, then the said state does not exist, effectively removed from reality. This is not a mathematical abstraction-it is the core function of collapse-based cosmology, where observational density defines realness.

We generalize this: collapse is not a projection onto a static subspace-it is a projection over a recursively updated basis of permissible configurations:

\begin{equation}
\hat{P}_t = \sum_i \ket{\phi_i(t)}\bra{\phi_i(t)}
\end{equation}

Where $\{\ket{\phi_i(t)}\}$ are time-dependent collapse-permissible eigenstates that adapt based on recursive field convergence.

\section{Field Evolution Equation}

We now introduce a comprehensive redefinition of the field evolution equation, integrating:
\begin{itemize}
  \item Space as a sieve: collapse defines rather than fills\cite{barbour_end_of_time}
  \item Dark flow potential: escape vectors across undefined gradients\cite{calzetta_hu_backreaction}
  \item Observational coherence-driven rapid expansion\cite{moffat_variable_c}
  \item Collapse-interdynamic fields and their interference\cite{pearle_collapse_quantized}
  \item Cosmological feedback from definitional vacuum stress\cite{penrose1996gravity}
  \item Refeeding of the real by the imaginary plane: recursive collapse as generative\cite{fewster_measurement_qft}
  \item Matter-antimatter ratio breakdown and collapse void seeding\cite{sakharov_1967}
\end{itemize}

\begin{equation}
\frac{\partial M(x,t)}{\partial t} = D \nabla^2 M - \eta \mathcal{R}(x,t) M + f(\theta) \cdot \rho_{\text{obs}}(x,t) + \Lambda_{\text{dark}}(x,t) - \nabla \cdot \vec{F}_{\text{flow}}(x,t) + \Phi_{\text{imag}}(x,t) + \Delta_{\text{AM}}(x,t)
\end{equation}

Where:
\begin{itemize}
  \item $\mathcal{R}(x,t)$ is the collapse curvature, contributing to field deformation
  \item $f(\theta)$ encodes imaginary-plane phase coherence amplification\cite{tHooft_determinism}
  \item $\Lambda_{\text{dark}}(x,t)$ is the localized collapse energy offset from dark matter/dark energy potential imbalance
  \item $\vec{F}_{\text{flow}}$ is the dark flow vector field emerging from collapse discontinuity and phase shear
  \item $\Phi_{\text{imag}}(x,t)$ is the feeding function from the imaginary to real plane, representing definitional reconstitution
  \item $\Delta_{\text{AM}}(x,t)$ is the collapse differential between matter and antimatter definitions
\end{itemize}

We define the imaginary feedforward field as:

\begin{equation}
\Phi_{\text{imag}}(x,t) = \int_\tau^t \Re\left[ e^{i \theta(x,\tau')} \cdot M_i(x,\tau') \right] d\tau'
\end{equation}

Where $M_i(x,t)$ is the imaginary component of the collapse field. This term models recursive redefinition, refeeding the real domain from the collapse residue.

Antimatter asymmetry is encoded via the definitional destruction factor:

\begin{equation}
\Delta_{\text{AM}}(x,t) = \left( \frac{42.3}{100} - \frac{57.7}{100} \right) \cdot M(x,t) \cdot \chi_{\text{annihilation}}(x,t)
\end{equation}

Where $\chi_{\text{annihilation}}(x,t)$ is the spatiotemporal annihilation index based on initial condition symmetry breaking. Regions of high $\Delta_{\text{AM}}$ experience collapse voiding-pockets where definition failed to stabilize.

These voids are observable as low-entropy zones where collapse memory has ruptured, mimicking underdense regions of large-scale structure.

We model $\Lambda_{\text{dark}}$ with a derived cosmological feedback relation:

\begin{equation}
\Lambda_{\text{dark}}(x,t) = \frac{\rho_{\text{DM}}(x,t) - \rho_{\text{DE}}(x,t)}{\rho_{\text{crit}}} \cdot \Theta(x,t)
\end{equation}

Where $\rho_{\text{DM}}$ and $\rho_{\text{DE}}$ are the local dark matter and dark energy densities, $\rho_{\text{crit}}$ is the field stability limit, and $\Theta(x,t)$ is the definitional alignment index (collapse coherence scalar).

The observational sieve property is expressed as:

\begin{equation}
\mathcal{S}_{\text{sieve}}(x,t) = \left(1 - \frac{\partial M}{\partial t}\bigg/ M \right) \cdot \nabla^2 M
\end{equation}

\section{Observer Effect and Measurement as a Field}

Measurement alters the state of a system-forcing it into either a particle or wave representation. This collapse behavior is the definition of a field influence~\cite{penrose1996gravity,pearle_collapse_quantized}.

Physics traditionally treats observation as passive. But measurement exerts causal structure; the system behaves differently because it’s being observed. This idea extends quantum theory into the field framework.

We define a collapse pressure:

\begin{equation}
C(x,t) = \frac{\partial M(x,t)}{\partial t} \cdot \nabla M(x,t)
\end{equation}

Collapse pressure becomes dominant when second-order recursion occurs. We refine this:

\begin{equation}
C(x,t) = \left( \nabla \cdot \vec{F}_M(x,t) + \frac{\partial^2 M}{\partial t^2} \right) \cdot \left| \nabla M(x,t) \right|
\end{equation}

Where $\vec{F}_M$ is the collapse flux field defined by observer reinforcement. This makes pressure a directional scalar in both space and time.

\section{Operator Formalism and Collapse Dynamics}

\begin{equation}
\langle 0 | \hat{P} \hat{H}_{\text{vac}} | 0 \rangle = \rho_{\text{virtual}}
\end{equation}

Here, $\hat{H}_{\text{vac}}$ is the Hamiltonian of vacuum modes, and $\rho_{\text{virtual}}$ is the net observable virtual energy based on collapse compatibility~\cite{fewster_measurement_qft}.

We propose a dynamic vacuum Hamiltonian:

\begin{equation}
\hat{H}_{\text{vac}}(t) = \sum_{k} E_k(t) \hat{a}^\dagger_k \hat{a}_k + \mathcal{C}_{\text{collapse}}(t)
\end{equation}

Where $\mathcal{C}_{\text{collapse}}$ encodes the observational backreaction. This breaks vacuum stationarity in the presence of recursive fields.

\section{Time Asymmetry and Collapse Directionality}

The collapse process is not time-reversible. Once observation defines a state, the informational entropy locked into that resolution cannot be reversed without external negation. Define the temporal collapse asymmetry:

\begin{equation}
\Delta T_C(x,t) = \frac{dS_C}{dt} - \frac{dS_C}{dt}\bigg|_{t \to -t}
\end{equation}

We refine this to include projection frequency:

\begin{equation}
\Delta T_C(x,t) = \int_0^t \left[ \Gamma_{\text{forward}}(t') - \Gamma_{\text{backward}}(t') \right] dt'
\end{equation}

Where $\Gamma_{\text{forward}}$ and $\Gamma_{\text{backward}}$ are directional projection densities tied to field definition bias.

\section{Imaginary Phase Locking and Collapse Orbitals}

Collapse does not occur on the real plane alone. Define the phase orbit in the complexified observer manifold:

\begin{equation}
\Psi(x,t) = A(x,t) e^{i \theta(x,t)}
\end{equation}

If $\theta(x,t)$ aligns periodically, we define a locked orbital of collapse frequency. Let the imaginary phase gradient determine stability:

\begin{equation}
\vec{\nabla}_\theta = \frac{\partial \theta}{\partial x} \hat{i} + \frac{\partial \theta}{\partial y} \hat{j} + \frac{\partial \theta}{\partial z} \hat{k}
\end{equation}

Now define a resonance function:

\begin{equation}
\Omega(x,t) = \sin^2\left(\theta(x,t) - \omega_0 t\right) \cdot M(x,t)
\end{equation}

Where $\omega_0$ is a resonance frequency of recursive collapse alignment.

\section{Topological Collapse Anchors}

Certain regions of space exhibit persistent definitional lock-in due to recursive observation geometry. Define an anchor point $x_a$ where:

\begin{equation}
\oint_{\partial V} M(x,t) dA = \text{constant}, \quad \forall t
\end{equation}

We extend this by defining a coherence shell:

\begin{equation}
\mathcal{S}_{\text{anchor}}(x,t) = \left\{ y \in \mathbb{R}^3 \mid |M(y,t) - M(x,t)| < \epsilon \right\}
\end{equation}

These collapse anchors act as nodal fixpoints in field topology. They may be observable as superheavy elements, neutron stars, or regions of abnormal cosmic coherence.



\nocite{chapter10_meta}
\printbibliography[title={Appendix I References}, keyword=chapter10]